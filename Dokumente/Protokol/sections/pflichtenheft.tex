\section{Über das Team}
\setauthor{Antonio Peric}
\begin{table}[h]
    \centering
    \caption{\textit{Tabelle 1.1: Informationen über den Bertreuer und Partner.}}
    \begin{tabular}{|p{0.5\textwidth}|p{0.5\textwidth}|}
        \hline
        Bertreuer & Prof. Mag. Ing. Hans Christian Hammer \\ \hline
        Partner & DI. Christian Aberger \\ \hline
    \end{tabular}
\end{table}

\section{Bertreuer und Partner}
\setauthor{Antonio Peric}
\begin{table}[h]
    \centering
    \caption{\textit{Tabelle 1.2: Informationen über das Projekt und das Team.}}
    \begin{tabular}{|p{0.5\textwidth}|p{0.5\textwidth}|}
        \hline
        Projektname & AGM - Abergymmobile \\ \hline
        Teamleiter & Antonio Kuvac \\ \hline
        Teammitglieder & Antonio Kuvac, Antonio Peric \\ \hline
        Erstellt am & 12.7.2023 \\ \hline
    \end{tabular}
\end{table}

\section{Ausgangssituation und Zielstellung}

\subsection{Ausgangssituation}
\setauthor{Antonio Peric}   
Aktuell ist der Prozess der Trainingsplangenerierung und -verwaltung im Fitnessstudio LionFit ineffizient und umständlich für Kundinnen und Kunden. Daher möchte das Fitnessstudio eine digitale Lösung implementieren, um den Prozess zu vereinfachen und die Effizienz zu steigern. Derzeit werden die Trainingspläne in einer Web-Applikation erstellt und als PDF ausgedruckt, während die Trainingsdaten auf dem ausgedruckten PDF erfasst werden. Dieser Prozess ist unpraktisch, da die Kundinnen und Kunden bei jeder Trainingseinheit ein ausgedrucktes PDF mitnehmen und die Daten manuell eingeben müssen.

\subsection{Zieldefinition}
\setauthor{Antonio Peric}  
Die Entwicklung einer nativen App für Android-Geräte zur Umsetzung von Trainingsplänen in einem Fitnessstudio gewinnt zunehmend an Bedeutung. Diese App ermöglicht es den Nutzer*innen, ihre individuellen Trainingspläne effektiv und effizient abzuarbeiten, 
\newline
\newline
Die Trainingsplanverwaltung stellt sicher, dass die Daten des Trainingsplans aktuell und auf die Bedürfnisse der*die Nutzer*innen zugeschnitten sind. Durch die Anbindung der App an die Trainingsplanverwaltung wird es ermöglicht, die relevanten Daten direkt auf dem Android-Gerät der*die Nutzer*innen abzurufen. Dies erleichtert die Organisation und Durchführung des Trainings, indem es den Nutzer*innen erlaubt, auf ihre individuellen Trainingspläne in Echtzeit zuzugreifen.
\newline
\newline
Am Ende einer Trainingssession werden die Trainingsdaten, wie beispielsweise die Anzahl der Wiederholungen, die Gewichte oder die Trainingsdauer, in die Trainingsdatenbank übertragen und gespeichert. Dies ermöglicht eine kontinuierliche Analyse und Anpassung der Trainingspläne, um eine optimale Unterstützung der*die Nutzer*innen in ihrer sportlichen Entwicklung zu gewährleisten.
\newline
\newline
Durch die Integration von Training und Datenverwaltung in einer nativen Android-App werden Fitnessstudios in die Lage versetzt, ein benutzerfreundliches, modernes und zielgerichtetes Trainingsumfeld für ihre Mitglieder*innen zu schaffen. Das trägt zur Steigerung der Motivation und der Erfolgschancen bei, da individuelle Trainingsziele leichter erreicht werden können.

\newpage
\subsection{Nicht Ziele}
\setauthor{Antonio Peric}  
Die Entwicklung einer App für Trainingspläne in Fitnessstudios birgt auch gewisse Herausforderungen und Risiken. Eine davon ist die Gefahr, eine zu komplex gestaltete App zu entwickeln, die den Nutzer*innen Schwierigkeiten bereitet und sie dazu veranlasst, stattdessen auf den traditionellen Zettel-Trainingsplan zurückzugreifen. Um dem entgegenzuwirken, sollte die App intuitiv und benutzerfreundlich gestaltet sein, sodass sie die Bedürfnisse der*die Nutzer*innen erfüllt und gleichzeitig den Trainingsprozess vereinfacht.
\newline
\newline
Ein weiteres Risiko besteht in der Entwicklung einer fehlerhaften App, die den Nutzer*innen Unannehmlichkeiten bereitet und ihre Trainingserfahrung beeinträchtigt. Um dies zu vermeiden, ist es wichtig, die App sorgfältig zu testen und mögliche Fehlerquellen frühzeitig zu identifizieren. Die Qualitätssicherung und regelmäßige Aktualisierung der App sind entscheidend für ihren Erfolg.
\newline
\newline
Schließlich kann auch das Design der App einen bedeutenden Einfluss auf die Akzeptanz bei den Nutzer*innen haben. Eine App, deren Design die Zielgruppe nicht anspricht, könnte weniger erfolgreich sein und das Potenzial der digitalen Trainingsplanunterstützung ungenutzt lassen. Daher ist es ratsam, bei der Gestaltung der App auf ansprechende und funktionale Designelemente zu achten, die die Nutzer*innen ansprechen und zum wiederholten Gebrauch motivieren.
\newline
\newline
Die Berücksichtigung dieser Herausforderungen bei der Entwicklung einer Trainingsplan-App ist essentiell, um eine positive Benutzererfahrung zu gewährleisten und den Nutzer*innen eine effektive und ansprechende Alternative zum traditionellen Zettel-Trainingsplan zu bieten.

\newpage
\section{Zielgruppe}
\setauthor{Antonio Peric}  
Diese Zielgruppe umfasst sowohl regelmäßige Fitnessstudio-Besucher*innen als auch Sportler*innen, 
die unabhängig von einem Fitnessstudio trainieren und eine digitale Lösung für die Verwaltung 
ihrer Trainingspläne suchen. Besonders praktisch für diese Zielgruppe ist, dass sie jederzeit 
Zugang zu ihren Trainingsplänen und -daten auf ihrem Smartphone haben. Dies bietet mehr Flexibilität 
und Übersicht bei der Gestaltung und Überwachung des Trainings. Außerdem müssen die Kunden*innen nicht mehr 
auf ausgedruckte Trainingspläne zurückgreifen und können stattdessen auf eine sichere und zuverlässige 
digitale Lösung setzen.

\section{Funktionale Anforderungen}
\setauthor{Antonio Peric}  
Die Benutzeroberfläche der nativen App sollte einfach und ansprechend gestaltet sein, um auch Personen, die nicht besonders computeraffin sind, eine leichte Handhabung zu ermöglichen. Darüber hinaus soll die Modernisierung des Trainingsplanprozesses die Kommunikation zwischen der Verwaltung und den Koordinatoren des Fitnessstudios LionFit vereinfachen und automatisieren.

\subsection{An die App}
\setauthor{Antonio Peric}  
Die effektive Nutzung einer Trainingsplan-App erfordert eine sorgfältige Planung und Umsetzung verschiedener Funktionen. Eine Möglichkeit, die Nutzer*innen in ihrem Trainingsprozess zu unterstützen, besteht darin, die Durcharbeitung der Trainingspläne als To-Do-Liste zu gestalten. Dies ermöglicht es den Nutzer*innen, ihre Fortschritte während des Trainings klar zu erkennen und ihre Motivation aufrechtzuerhalten.
\newline
\newline
Nach Abschluss jeder Übung sollte die App die Möglichkeit bieten, den Trainingsplan zu überprüfen und gegebenenfalls anzupassen. Dies ermöglicht eine individuelle Anpassung des Trainings und stellt sicher, dass die Nutzer*innen stets auf dem aktuellen Stand ihrer Trainingsziele sind. Die flexible Anpassung des Trainingsplans trägt dazu bei, die Effektivität des Trainings zu erhöhen und den Bedürfnissen der*die Nutzer*innen gerecht zu werden.
\newpage
Die Verfügbarkeit des überarbeiteten Trainingsplans zu jeder Zeit ist ein weiterer wichtiger Aspekt, der es den Nutzer*innen ermöglicht, ihre Trainingspläne erneut durchzuarbeiten und an ihren Zielen kontinuierlich zu arbeiten. Durch die ständige Verfügbarkeit der aktualisierten Trainingspläne können die Nutzer*innen ihre Trainingsfortschritte effizient verfolgen und die erforderlichen Anpassungen vornehmen.
\newline
\newline
Schließlich sollte die App auch eine Historie der alten Trainingspläne aufbewahren, um den Nutzer*innen einen Überblick über ihre Trainingsentwicklung und die Möglichkeit, auf vergangene Trainingspläne für zukünftige Referenzen zuzugreifen, zu bieten. Dies kann als wertvolles Instrument für die Selbstreflexion und Analyse des Trainingsfortschritts dienen.

\section{App}
\subsection{Allgemeine Beschreibung}
\setauthor{Antonio Peric}  
Mit unserer innovativen mobilen Anwendung bieten wir Ihnen die Möglichkeit, Ihre individuellen Trainingspläne bequem und effizient über Ihr Smartphone abzuarbeiten. Die Anwendung präsentiert den Nutzer*innen die Trainingspläne in Form einer leicht verständlichen To-Do-Liste, die eine strukturierte und systematische Durchführung jeder Übung gewährleistet. Dies fördert die Trainingsdisziplin und unterstützt die Nutzer*innen dabei, ihre persönlichen Trainingsziele zu erreichen.
\newline
\newline
Nach Abschluss jeder Übung haben die Nutzer*innen die Möglichkeit, ihren Trainingsplan zu überprüfen und gegebenenfalls anzupassen. Diese Flexibilität ermöglicht es, das Training kontinuierlich an die individuellen Bedürfnisse und Fortschritte der*die Nutzer*innen anzupassen, um eine optimale Trainingsgestaltung sicherzustellen.
\newline
\newline
Der überarbeitete Trainingsplan ist jederzeit für die Nutzer*innen zugänglich, sodass sie ihn bei Bedarf erneut durcharbeiten können. Dies gewährleistet eine hohe Verfügbarkeit der Trainingsinformationen und erleichtert die Planung und Durchführung des Trainings im Alltag der*die Nutzer*innen.
\newpage
Darüber hinaus werden alte Trainingspläne in der Anwendungshistorie gespeichert, sodass sie bei Bedarf als Referenz herangezogen werden können. Die Archivierung der Trainingshistorie ermöglicht den Nutzer*innen, ihren Trainingsfortschritt im Laufe der Zeit zu analysieren und eventuelle Anpassungen oder Modifikationen ihrer Trainingsziele vorzunehmen.