\section{Über das Team}
\setauthor{Antonio Peric}
\begin{table}[h]
    \centering
    \caption{\textit{Tabelle 1.1: Informationen über den Bertreuer und Partner.}}
    \begin{tabular}{|p{0.5\textwidth}|p{0.5\textwidth}|}
        \hline
        Bertreuer & Prof. Mag. Ing. Hans Christian Hammer \\ \hline
        Partner & DI. Christian Aberger \\ \hline
    \end{tabular}
\end{table}

\section{Bertreuer und Partner}
\setauthor{Antonio Peric}
\begin{table}[h]
    \centering
    \caption{\textit{Tabelle 1.2: Informationen über das Projekt und das Team.}}
    \begin{tabular}{|p{0.5\textwidth}|p{0.5\textwidth}|}
        \hline
        Projektname & AGM - Abergymmobile \\ \hline
        Teamleiter & Antonio Kuvac \\ \hline
        Teammitglieder & Antonio Kuvac, Antonio Peric \\ \hline
        Erstellt am & 12.7.2023 \\ \hline
    \end{tabular}
\end{table}

\section{Ausgangssituation und Zielstellung}

\subsection{Ausgangssituation}
\setauthor{Antonio Peric}   
Aktuell ist der Prozess der Trainingsplangenerierung und -verwaltung im 
Fitnessstudio LionFit ineffizient und umständlich für Kunden. 
Daher möchte das Fitnessstudio eine digitale Lösung implementieren,
um den Prozess zu vereinfachen und die Effizienz zu steigern. 
Derzeit werden die Trainingspläne in einer Web-Applikation 
erstellt und als PDF ausgedruckt, während die Trainingsdaten 
auf dem ausgedruckten PDF erfasst werden. Dieser Prozess ist 
unpraktisch, da die Kunden bei jeder Trainingseinheit ein 
ausgedrucktes PDF mitnehmen und die Daten manuell eingeben müssen.

\subsection{Zieldefinition}
\setauthor{Antonio Peric}  
\begin{itemize}
    \item Entwicklung einer nativen App für Android Devices zur Abarbeitung von Trainingsplänen in einem Fitnessstudio.    
    \item Die Daten des Trainingsplans werden aus der Datenbank der Trainingsplanverwaltung ausgelesen. 
    \item Am Ende der Trainingssession wird die Trainingsdatenbank mit den Trainingsdaten ergänzt.
\end{itemize}

\subsection{Nicht Ziele}
\setauthor{Antonio Peric}  
\begin{itemize}
    \item Entwicklung einer zu komplex gestalteten App die zum Verwenden vom Zettel Trainingsplan anregt.    
    \item Entwicklung einer sehr fehlerhaften App.
    \item Entwicklung einer App dessen Design die Benutzer nicht anspricht.
\end{itemize}

\section{Zielgruppe}
\setauthor{Antonio Peric}  
Diese Zielgruppe umfasst sowohl regelmäßige Fitnessstudio-Besucher als auch Sportler, 
die unabhängig von einem Fitnessstudio trainieren und eine digitale Lösung für die Verwaltung 
ihrer Trainingspläne suchen. Besonders praktisch für diese Zielgruppe ist, dass sie jederzeit 
Zugang zu ihren Trainingsplänen und -daten auf ihrem Smartphone haben. Dies bietet mehr Flexibilität 
und Übersicht bei der Gestaltung und Überwachung des Trainings. Außerdem müssen die Kunden nicht mehr 
auf ausgedruckte Trainingspläne zurückgreifen und können stattdessen auf eine sichere und zuverlässige 
digitale Lösung setzen.

\newpage
\section{Funktionale Anforderungen}
\setauthor{Antonio Peric}  
Die Benutzeroberfläche der native App soll eine einfache und 
ansprechende Gestaltung aufweisen, um auch nicht computeraffinen Personen eine leichte 
Handhabung zu ermöglichen. Zusätzlich soll durch die Modernisierung des Trainingsplanprozesses 
die Kommunikation zwischen Verwaltung und Koordinatoren im Fitnessstudio LionFit vereinfacht 
und automatisiert werden.

\subsection{An die App}
\setauthor{Antonio Peric}  
\begin{itemize}
    \item Durcharbeitung der Trainingspläne als To-Do-Liste.
    \item Überprüfung und Anpassung des Trainingsplans nach Abschluss jeder Übung.
    \item Verfügbarkeit des überarbeiteten Trainingsplans zu jeder Zeit für eine erneute Durcharbeitung.
    \item Aufbewahrung des alten Trainingsplans in der Historie für zukünftige Referenzen.
\end{itemize}

\section{App}


\subsection{Allgemeine Beschreibung}
\setauthor{Antonio Peric}  
Mit unserer mobilen Anwendung können Sie Ihre Trainingspläne bequem über Ihr 
Smartphone durcharbeiten. Die Anwendung präsentiert Ihnen Ihre Trainingspläne 
in Form einer To-Do-Liste, damit Sie jede Übung sorgfältig durchführen können. 
Nach Abschluss jeder Übung können Sie Ihren Trainingsplan überprüfen und anpassen, 
um sicherzustellen, dass Sie optimal vorankommen. Der überarbeitete Trainingsplan 
steht Ihnen jederzeit zur Verfügung, damit Sie ihn erneut durcharbeiten können. 
Darüber hinaus wird Ihr alter Trainingsplan in der Historie gespeichert, damit Sie 
ihn später als Referenz nutzen können.



